\documentclass[]{article}
\usepackage{lmodern}
\usepackage{amssymb,amsmath}
\usepackage{ifxetex,ifluatex}
\usepackage{fixltx2e} % provides \textsubscript
\ifnum 0\ifxetex 1\fi\ifluatex 1\fi=0 % if pdftex
  \usepackage[T1]{fontenc}
  \usepackage[utf8]{inputenc}
\else % if luatex or xelatex
  \ifxetex
    \usepackage{mathspec}
  \else
    \usepackage{fontspec}
  \fi
  \defaultfontfeatures{Ligatures=TeX,Scale=MatchLowercase}
\fi
% use upquote if available, for straight quotes in verbatim environments
\IfFileExists{upquote.sty}{\usepackage{upquote}}{}
% use microtype if available
\IfFileExists{microtype.sty}{%
\usepackage{microtype}
\UseMicrotypeSet[protrusion]{basicmath} % disable protrusion for tt fonts
}{}
\usepackage[margin=1in]{geometry}
\usepackage{hyperref}
\hypersetup{unicode=true,
            pdftitle={w203\_Lab3\_feedback\_to\_Group1},
            pdfauthor={Joanna Wang, Douglas Xu},
            pdfborder={0 0 0},
            breaklinks=true}
\urlstyle{same}  % don't use monospace font for urls
\usepackage{graphicx,grffile}
\makeatletter
\def\maxwidth{\ifdim\Gin@nat@width>\linewidth\linewidth\else\Gin@nat@width\fi}
\def\maxheight{\ifdim\Gin@nat@height>\textheight\textheight\else\Gin@nat@height\fi}
\makeatother
% Scale images if necessary, so that they will not overflow the page
% margins by default, and it is still possible to overwrite the defaults
% using explicit options in \includegraphics[width, height, ...]{}
\setkeys{Gin}{width=\maxwidth,height=\maxheight,keepaspectratio}
\IfFileExists{parskip.sty}{%
\usepackage{parskip}
}{% else
\setlength{\parindent}{0pt}
\setlength{\parskip}{6pt plus 2pt minus 1pt}
}
\setlength{\emergencystretch}{3em}  % prevent overfull lines
\providecommand{\tightlist}{%
  \setlength{\itemsep}{0pt}\setlength{\parskip}{0pt}}
\setcounter{secnumdepth}{0}
% Redefines (sub)paragraphs to behave more like sections
\ifx\paragraph\undefined\else
\let\oldparagraph\paragraph
\renewcommand{\paragraph}[1]{\oldparagraph{#1}\mbox{}}
\fi
\ifx\subparagraph\undefined\else
\let\oldsubparagraph\subparagraph
\renewcommand{\subparagraph}[1]{\oldsubparagraph{#1}\mbox{}}
\fi

%%% Use protect on footnotes to avoid problems with footnotes in titles
\let\rmarkdownfootnote\footnote%
\def\footnote{\protect\rmarkdownfootnote}

%%% Change title format to be more compact
\usepackage{titling}

% Create subtitle command for use in maketitle
\newcommand{\subtitle}[1]{
  \posttitle{
    \begin{center}\large#1\end{center}
    }
}

\setlength{\droptitle}{-2em}

  \title{w203\_Lab3\_feedback\_to\_Group1}
    \pretitle{\vspace{\droptitle}\centering\huge}
  \posttitle{\par}
    \author{Joanna Wang, Douglas Xu}
    \preauthor{\centering\large\emph}
  \postauthor{\par}
      \predate{\centering\large\emph}
  \postdate{\par}
    \date{November 27, 2018}


\begin{document}
\maketitle

\begin{enumerate}
\def\labelenumi{\arabic{enumi}.}
\item
  Introduction \tab the introduction is very clear with background
  information + research question + identified parameters. However,
  maybe the research question should be asked in a more open format? So
  that you leave enough room for other parameters that you may find
  strongly correlated in the analysis
\item
  The initial data loading and cleaning: \tab the data cleaning is very
  well executed on the variables of key interest. Also, anomalies are
  identified, but you are being very careful about removing anything
  that you do not believe is error. \tab one thing that might be worth
  thinking about is that, should identifying key variables come before
  data cleaning or after. It might make more logical sense to make
  decision on the key explanatory variables after cleaning the dataset
  initially
\item
  Model building process: \tab very good univariate analysis on each
  explanatory variable and outcome variable. \tab \textbf{a few
  suggestions}: \tab - in the EDA, analyze several more variables \tab -
  more explanation on why are those explanatory variables chosen over
  the rest of dataset \tab - use potential transformations to variables
  that are not close to normal distribution
\item
  Regression model: \tab nice detailed reporting on model coefficients,
  and plotting of standard errors \tab \textbf{a few suggestions}:
  \tab - more detailed discussion of the coefficients from each model,
  and what conclusions can be drawn from the model \tab - more
  discussion on how the coefficients are complying with the 6 CLM
  assumptions \tab - more detailed discussion between each model,
  especially in how the third model helps in verifying the validity of
  previous two models \tab - compare the model coefficient in a table
\item
  Omitted Variable Discussion: \tab according to the question and answer
  in piazza, omitted variable seems to be variable that is outside of
  the dataset. Therefore, there should be some discussion on some
  important variable that are important for the model, but not present
  in the dataset
\end{enumerate}

\subsubsection{Several small things}\label{several-small-things}

\begin{enumerate}
\def\labelenumi{\arabic{enumi}.}
\tightlist
\item
  the units in \(avgsen\) is actually days
\end{enumerate}


\end{document}
